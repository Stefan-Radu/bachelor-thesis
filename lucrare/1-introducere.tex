\chapter{Introducere}

\section{Context}

\emph{Meltodwn} si \emph{Spectre} fac parte din clasa larga a Atacurilor de tip
\emph{Side-Channel}, care exploateaza mai degraba efecte secundare rezultate
din implementarea unui sistem, decat defecte in implementarea algoritmilor ce
ruleaza pe acel sistem. De-a lungul timpului ai aparut numeroase atacuri de tip
\emph{Side-channel}:

\begin{itemize}
  \item Timing attacks. Aceste atacuri au ca scop compromiterea unui sistem prin
    intermediul analizei statistice a timpilor de executie ai unui algoritm pe   
    diverse seturi de date de intrare (eg. compromiterea unui sistem de criptare
    RSA la distanta \cite{timing_practical})
  \item Cache attacks. Aceste atacuri se bazeaza pe abilitatea unui atacator
    a monitoriza accesarile victimei a unor zone de memorie partajate si deducerea
    unor concluzii din modul in care aceste accesari influenteaza cache-ul 
    procesorului. In cazuri extreme s-a demonstrat ca se pot divulga chei criptografice 
    secrete prin intermediul acestor tehnici \cite{percival2005cache}.
  \item Data remanence attacks. Aceste atacuri implica accesul asupra unor date
    dupa o presupusa stergere a acestora in prealabil. Un exemplu clar este 
    atacul de tip \emph{Cold Boot} \cite{cold_boot} in care un atacator cu
    acces fizic la o masina paoate citi intreaga memorie RAM dupa efectuarea
    unui resetari a computerului.
  \item Rowhammer are un loc special in clasa de atacuri de tip \emph{Side-Channel}. 
    Prin accesul repetat al unei zone de memorie s-a observat ca incarcatura elecatrica
    poate afecta zonele adiacente, provocand scurgeri de informatie. Pe baza acestei
    tehnici s-au putut construi atacuri de tip escalare de privilegii \cite{rowhammer}.
  \item Mai exista si alte tipuri de atacuri care exploateaza consumul energetic,
    campul electromagnetic generat de componentele electronice, sau chiar si sunetul
    generat de sistem.
\end{itemize}

\emph{Meltodwn} si \emph{Spectre} se folosesc de idei asemanatoare cu cele
mentionate in \emph{Timing Attacks} si in \emph{Cache Attacks} pentru a crea un
canal de comunicare ascuns. Pe acest canal se transmit informatii accesate in
mod malititos prin intermediul unor hibe in implementarea la nivel hardware a
compunterelor modern. Se va descrie cum exploatarea acestor defecte este facuta
posibila prin intermediul executiei speculative in cadrul procesorului.
Executia speculativa prcare esupune executia in avans a instructiunuilor pentru
a salva timpi morti si a imbunatati performanta. Fluxul de executie poate fi
manipulat in asa fel incat speculativ sa se execute instructiuni care nu s-ar
executa vreodata in cadrul executiei normale a programului. Pentru mentinerea
consistentei si corectitudinii rezultatelor obitnute in urma executiei
algoritmului, rezultatele instructiunilor executate speculativ in mod eronat sunt
omise, iar starea interna este resetata. In momentul resetarii, starea cache-ului
nu este si ea resetata, iar acest fapt poate fi exploatat, iar informatiile obtinute
transmise printr-un \emph{side-channel}.

\section{Motivatia Personala}

Din fire, incerc mereu sa aprofundez cat mai in amanunt subiectele care ma
intereseaza, pentru a intelege in profunzime. Natural -- consider eu -- am
ajuns atras de subdomeniul securitatii care presupune o intelegere a sistemelor
de informatii. Dintre atacurile cibernetice, \emph{Spectre} si \emph{Meltdown}
mi-au starnit interesul pe deoparte prin rezultatele remarcabile obtinute si pe
de alta parte prin nivelul ridicat de subtilitate al vulnerabilitatilor
exploatate.

\section{Scopul Lucrarii}

Lucrarea de fata este rezultatul studiului personal al materialelor originale
care expuneau aceste atacuri. Aceasta are drept scop explicarea clara, dar
succinta a \emph{Meltdown} si \emph{Spectre}, intr-un mod accesibil cititorilor
interesati, dar nu neaparat neavizati. Va fi de asemenea explicata o
implementare cu scop demonstrativ in care un proces neprivilegiat citeste in
mod neautorizat zone de memorie dintr-un proces victima, prin intermediul
tehicilor descrise pe parcurs.

\section{Structura Lucrarii}

Lucrarea este impartita in urmatoarele capitole:

\begin{enumerate}
  \item Introducere -- se prezinta o viziune de ansamblu asupra atacurilor
    ce urmeaza a se fie prezentate,un mic istoric al tehiniclor si motivatia
    personala pentru realizarea acestei lucrari.
  \item Preliminarii -- se prezinta notiuni de \emph{Sisteme de Operare},
    \emph{Arhitectura Sistemelor de Calcul} si \emph{Securitate} relevante
    intelegerii atacurilor discutate.
  \item Atacul Meltdown -- se discuta detalii de functionalitate, metode de mitigare
    si detalii de reproducere a atacul meltdown.
  \item Atacuri Spectre -- se ilustreaza deosebirile fata de Meltdown, precum si
    detalii de specifice de functionalitate pentru cele doua variante principale ale
    Spectre. Sunt prezentate metode de mitigare si starea actuala a
    vulnerabilitatilor.
  \item POC Spectre -- se prezinta o implementare demonstrativa a \emph{Spectre v1}
  \item Concluzii -- se sumarizeaza cele discutate pe parcurs si mentioneaza
    directiile viitoare
\end{enumerate}
