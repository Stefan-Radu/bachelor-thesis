% Sablon pentru realizarea lucrarii de licenta, conform cu recomandarile
% din ghidul de redactare:
% - https://fmi.unibuc.ro/finalizare-studii/
% - https://drive.google.com/file/d/1xj9kZZgTkcKMJkMLRuoYRgLQ1O8CX0mv/view

% Multumiri lui Gabriel Majeri, acest sablon a fost creat pe baza
% codului sursa a lucrarii sale de licenta. 
% Codul sursa: https://github.com/GabrielMajeri/bachelors-thesis
% Website: https://www.gabrielmajeri.ro/
%
% Aceast sablon este licentiat sub Creative Commons Attribution 4.0 International License.

\documentclass{report}[12pt, a4paper]

% Suport pentru diacritice și alte simboluri
\usepackage{fontspec}

% Suport pentru mai multe limbi
\usepackage{polyglossia}

% Setează limba textului la română
\setdefaultlanguage{romanian}
% Am nevoie de engleză pentru rezumat
\setotherlanguages{english}

% Indentează și primul paragraf al fiecărei noi secțiuni
\SetLanguageKeys{romanian}{indentfirst=true}

% Suport pentru diferite stiluri de ghilimele
\usepackage{csquotes}

\DeclareQuoteStyle{romanian}
  {\quotedblbase}
  {\textquotedblright}
  {\guillemotleft}
  {\guillemotright}

% Utilizează biblatex pentru referințe bibliografice
\usepackage[
    maxbibnames=50,
    sorting=nty,
    backend=biber,
    style=numeric
]{biblatex}

\addbibresource{bibliography.bib}

% Setează spațiere inter-linie la 1.5
\usepackage{setspace}
\onehalfspacing

% Modificarea geometriei paginii
\usepackage{geometry}

% Include funcțiile de grafică
\usepackage{graphicx}
% Încarcă imaginile din directorul `images`
\graphicspath{{./images/}}

% Listări de cod
\usepackage{listings}

% Linkuri interactive în PDF
\usepackage[
    colorlinks,
    linkcolor={black},
    menucolor={black},
    citecolor={black},
    urlcolor={blue}
]{hyperref}

% Simboluri matematice codificate Unicode
\usepackage[warnings-off={mathtools-colon,mathtools-overbracket}]{unicode-math}

% Comenzi matematice
\usepackage{amsmath}
\usepackage{mathtools}

% Formule matematice
\newcommand{\bigO}[1]{\symcal{O}\left(#1\right)}
\DeclarePairedDelimiter\abs{\lvert}{\rvert}

% Suport pentru rezumat în două limbi
% Bazat pe https://tex.stackexchange.com/a/70818
\newenvironment{abstractpage}
  {\cleardoublepage\vspace*{\fill}\thispagestyle{empty}}
  {\vfill\cleardoublepage}
\renewenvironment{abstract}[1]
  {\bigskip\selectlanguage{#1}%
   \begin{center}\bfseries\abstractname\end{center}}
  {\par\bigskip}

% Suport pentru anexe
\usepackage{appendix}

% Stiluri diferite de headere și footere
\usepackage{fancyhdr}

\fancypagestyle{front}{
  \fancyhf{}
  \renewcommand{\headrulewidth}{0pt}
  \cfoot{}
}
\fancypagestyle{main}{
  \fancyhf{}
  \renewcommand\headrulewidth{0pt}
  \fancyhead[C]{}
  \fancyfoot[C]{\thepage}
}

% Metadate
\title{Atacuri Speculative}
\author{Radu Ștefan-Octavian}

% Generează variabilele cu @
\makeatletter

\begin{document}

% Front matter
\cleardoublepage
\pagestyle{front}
\let\ps@plain\ps@front

% Pagina de titlu
\include{0-title}

\restoregeometry

\addtocounter{page}{1}

% Rezumatul
\begin{abstractpage}

\begin{abstract}{romanian}

  Computerele moderne folosesc tehnici de optimizare precum \emph{executie
  out-of-order} si \emph{branch prediction}. \emph{Meltdown} si \emph{Spectre}
  sunt doua atacuri care exploateaza efectele secundare aparute la nivel
  microarhitectural in urma optimizarilor mentionate. Prin intermediul acestora
  un atacator poate citi date private din zone arbitrare din memorie, fara
  privilegii si fara a exploata niciun bug de natura software. \emph{Intel},
  \emph{AMD} si \emph{ARM} au fost fortate in urma divulgarii acestor atacuri
  sa isi schimbe designul procesoarelor in incercarea de a mitiga
  vulnerabilitatile la nivel hardware. In ciuda solutiilor implementate, la
  jumatatea anului 2022, \emph{Spectre} afecteaza in continure majoritatea
  computerelor din lumea intreaga si ramane un pericol pentru utilizatori si un
  subiect de mare interes pentru cercetatori. In aceasta lucrare vor fi
  prezentate particularitatile celor doua atacuri, si o implementare
  demonstrativa a unui atac de tip \emph{Spectre}.

\end{abstract}

\begin{abstract}{english}
  
  Modern computers are equiped with features such as \emph{out-of-order
  execution} and \emph{branch prediction}, which are used to reduce CPU idel
  time and improve performance. \emph{Meltdown} and \emph{Spectre} are to cyber
  attacks that exploit microarhitectural side-effects which apper as a result
  of such optimization techiniques being used. An attacker can read private
  data of the vicim at arbitrary locations in memory, without exploiting any
  software bug. \emph{Intel}, \emph{AMD} and \emph{ARM} were forced to redesign
  their CPUs in order to migiate the risks posed by \emph{Meltdown} and
  \emph{Spectre}. Despite deployed mitigations, in the second half of 2022,
  most computers in the world are vulnerable to variations of \emph{Spectre}
  attacks, billions of users begin at risk. This class of attacks remains a
  subject of great interest for researchers in the field of security. In this
  work, the technicalities and implications of both attacks will be covered.
  Moreover, a proof of concept for a \emph{Spectre} attack will be presented.

\end{abstract}

\end{abstractpage}


\tableofcontents

% Main matter
\cleardoublepage
\pagestyle{main}
\let\ps@plain\ps@main

\chapter{Introducere}

\section{Context}

\emph{Meltodwn} si \emph{Spectre} fac parte din clasa larga a Atacurilor de tip
\emph{Side-Channel}, care exploateaza mai degraba efecte secundare rezultate
din implementarea unui sistem, decat defecte in implementarea algoritmilor ce
ruleaza pe acel sistem. De-a lungul timpului ai aparut numeroase atacuri de tip
\emph{Side-channel}:

\begin{itemize}
  \item Timing attacks. Aceste atacuri au ca scop compromiterea unui sistem prin
    intermediul analizei statistice a timpilor de executie ai unui algoritm pe   
    diverse seturi de date de intrare (eg. compromiterea unui sistem de criptare
    RSA la distanta \cite{timing_practical})
  \item Cache attacks. Aceste atacuri se bazeaza pe abilitatea unui atacator
    a monitoriza accesarile victimei a unor zone de memorie partajate si deducerea
    unor concluzii din modul in care aceste accesari influenteaza cache-ul 
    procesorului. In cazuri extreme s-a demonstrat ca se pot divulga chei criptografice 
    secrete prin intermediul acestor tehnici \cite{percival2005cache}.
  \item Data remanence attacks. Aceste atacuri implica accesul asupra unor date
    dupa o presupusa stergere a acestora in prealabil. Un exemplu clar este 
    atacul de tip \emph{Cold Boot} \cite{cold_boot} in care un atacator cu
    acces fizic la o masina paoate citi intreaga memorie RAM dupa efectuarea
    unui resetari a computerului.
  \item Rowhammer are un loc special in clasa de atacuri de tip \emph{Side-Channel}. 
    Prin accesul repetat al unei zone de memorie s-a observat ca incarcatura elecatrica
    poate afecta zonele adiacente, provocand scurgeri de informatie. Pe baza acestei
    tehnici s-au putut construi atacuri de tip escalare de privilegii \cite{rowhammer}.
  \item Mai exista si alte tipuri de atacuri care exploateaza consumul energetic,
    campul electromagnetic generat de componentele electronice, sau chiar si sunetul
    generat de sistem.
\end{itemize}

\emph{Meltodwn} si \emph{Spectre} se folosesc de idei asemanatoare cu cele
mentionate in \emph{Timing Attacks} si in \emph{Cache Attacks} pentru a crea un
canal de comunicare ascuns. Pe acest canal se transmit informatii accesate in
mod malititos prin intermediul unor hibe in implementarea la nivel hardware a
compunterelor modern. Se va descrie cum exploatarea acestor defecte este facuta
posibila prin intermediul executiei speculative in cadrul procesorului.
Executia speculativa prcare esupune executia in avans a instructiunuilor pentru
a salva timpi morti si a imbunatati performanta. Fluxul de executie poate fi
manipulat in asa fel incat speculativ sa se execute instructiuni care nu s-ar
executa vreodata in cadrul executiei normale a programului. Pentru mentinerea
consistentei si corectitudinii rezultatelor obitnute in urma executiei
algoritmului, rezultatele instructiunilor executate speculativ in mod eronat sunt
omise, iar starea interna este resetata. In momentul resetarii, starea cache-ului
nu este si ea resetata, iar acest fapt poate fi exploatat, iar informatiile obtinute
transmise printr-un \emph{side-channel}.

\section{Motivatia Personala}

Din fire, incerc mereu sa aprofundez cat mai in amanunt subiectele care ma
intereseaza, pentru a intelege in profunzime. Natural -- consider eu -- am
ajuns atras de subdomeniul securitatii care presupune o intelegere a sistemelor
de informatii. Dintre atacurile cibernetice, \emph{Spectre} si \emph{Meltdown}
mi-au starnit interesul pe deoparte prin rezultatele remarcabile obtinute si pe
de alta parte prin nivelul ridicat de subtilitate al vulnerabilitatilor
exploatate.

\section{Scopul Lucrarii}

Lucrarea de fata este rezultatul studiului personal al materialelor originale
care expuneau aceste atacuri. Aceasta are drept scop explicarea clara, dar
succinta a \emph{Meltdown} si \emph{Spectre}, intr-un mod accesibil cititorilor
interesati, dar nu neaparat neavizati. Va fi de asemenea explicata o
implementare cu scop demonstrativ in care un proces neprivilegiat citeste in
mod neautorizat zone de memorie dintr-un proces victima, prin intermediul
tehicilor descrise pe parcurs.

\section{Structura Lucrarii}

Lucrarea este impartita in urmatoarele capitole:

\begin{enumerate}
  \item Introducere -- se prezinta o viziune de ansamblu asupra atacurilor
    ce urmeaza a se fie prezentate,un mic istoric al tehiniclor si motivatia
    personala pentru realizarea acestei lucrari.
  \item Preliminarii -- se prezinta notiuni de \emph{Sisteme de Operare},
    \emph{Arhitectura Sistemelor de Calcul} si \emph{Securitate} relevante
    intelegerii atacurilor discutate.
  \item Atacul Meltdown -- se discuta detalii de functionalitate, metode de mitigare
    si detalii de reproducere a atacul meltdown.
  \item Atacuri Spectre -- se ilustreaza deosebirile fata de Meltdown, precum si
    detalii de specifice de functionalitate pentru cele doua variante principale ale
    Spectre. Sunt prezentate metode de mitigare si starea actuala a
    vulnerabilitatilor.
  \item POC Spectre -- se prezinta o implementare demonstrativa a \emph{Spectre v1}
  \item Concluzii -- se sumarizeaza cele discutate pe parcurs si mentioneaza
    directiile viitoare
\end{enumerate}

\chapter{Preliminarii}

\section{Out-of-order Execution \& Instructiuni Tranzitorii}

In trecut procesoarele executau instructiunile in ordinea in care acestea erau
preluate de la compilator, cate una pe rand. In multe situatii instructiuni mai
costisitoare blocau fluxul de executie, iar procesorul devenea partial inactiv.
Procesoarele moderne se folosesc de o serie de tehnici grupate sub umbrela
\emph{Out-of-order Execution}, introduse pentru prima data la mijlocul anilor
1990 \cite{what_is_speculative_execution}, in urma unui algoritm dezvoltat de 
Tomasulo in 1967 \cite{tomasulo1967} care permitea programarea dinamica
a ordinii instructiunilor si alocarea acestora pe mai multe unitati de executie
care ruleaza in paralel. Scopul acestei tehnici este utilizarea exhaustiva a
resurselor disponibile pe procesor, pentru cresterea performantei. 

Aceasta optimizare duce la situatii in care unele instructiuni executate trebuie 
respinse, iar starea programului intoarsa la una anterioara (din cauza decansarii 
unei exceptii in urma accesarii unei zone de memorie interzisa de exemplu). Aceste 
tipuri de instructiuni numite in continuare \emph{Instructiuni Tranzitorii} stau la 
baza atacului \emph{Meltdown} \cite{meltdown2018}.


\section{Branch Prediction \& Executie Speculativa}

Pe baza \emph{Branch Processing Unit} (\emph{BPU}) din interiorul procesoarelor
moderne incearca sa prezica, in cazul unei ramificari (\emph{if}), sau final de
iteratie (\emph{for, while}), ramura corecta pe care va fi urmata. In cazul in
care fluxul de executie stagneaza la un astfel punct de bifurcare (de exemplu,
in asteptarea incarcarii din memorie a valorii unei variabile), se poate folosi
prezicerea data de \emph{BPU} pentru a executa speculativ instructiunile
urmatoare. Dupa ce executia instructiunii care decide bifurcarea este finalizata
rezultatele obtinute speculativ sunt fie pastrate fie respinse \cite{spectre2019}.

Branch prediction are in general o acuratete foarte ridicata, chiar de peste $95\%$ \cite{what_is_speculative_execution}, asadar executand speculativ s-au obtinut
imbunatatiri considerabile de performanta. Cu toate acestea, in cazurile in care
ramura de executie nu este prezisa corect, se vor executa instructiuni care nu ar
fi avut loc in cadrul executie secventiale, \emph{in-order execution}. Bineinteles,
aceste instructiuni vor fi \emph{rolled-back}, iar rezultatul final va fi cel asteptat,
dar la nivel micro-arhitectural se pot observa si masura niste efecte neprevazute ale
acestor instructiuni executate \emph{out-of-order}. Analizarea cu grija a acestor efecte
secundare sta la baza atacurilor de tip \emph{Spectre} \cite{spectre2019}.


\section{CPU Cache}

Deoarece incarcarea valorilor din memoria RAM in cpu este foarte costisitoare, in 
cadrul procesoarelor exista niste zone de memorie foarte rapide, de dimensiuni reduse, 
ce poarta denumirea de emph{cache-uri}. Acestea retin valorile folosite cel mai des 
intr-un anumit interval de timp. Prin retinerea si citirea valorilor din cache,
se mascheaza incarcarea initial relativ lenta si se castiga timp pretios de executie.


\subsection{Atacuri asupra memoriei cache}

Deoarece memoria cache este mult mai rapida, prin intermediul unui ceas de mare precizie
putem distinge intre accesare din memorie si accesarea din \emph{cache} a unei variabile.
Sa consideram urmatorul exemplu:

\begin{lstlisting}[language=c]
  uint32_t value = 10;
  addr = &value;

  time = __rdtscp(&junk);
  junk = *addr;
  // prima accesare din memorie
  memory_time = __rdtscp(&junk) - time;

  addr = &value;
  time = __rdtscp(&junk);
  junk = *addr;
  // a doua accesare din cache
  cache_time = __rdtscp(&junk) - time;
\end{lstlisting}


Timpul de accesare al valorii corespunzatoare variabilei \textbf{value} poate
fi calculat utilizand instructiunea \texttt{\_\_rdtscp} specifica procesoarelor
Intel. Aceasta permite citirea \emph{time-stamp counter-ului} din procesor
\cite{rdtscp}. Prin doua masuratori ce incadreaza dereferentierea pointer-ului
catre \texttt{value}, putem masura numarul de ciclii de procesor necesari
operatiei. Repetand experimentul de $10000$ de ori si calcularea mediei
timpului de acces pentru fiecare caz, obtin urmatoarele:

\begin{itemize}
  \setlength\itemsep{0em}
  \item incarcarea din memorie dureaza aproximativ $250$ de ciclii
  \item incarcarea din cache dureaza aproximativ $23$ de ciclii
\end{itemize}

Diferente considerabile precum acestea sunt exploatate in cadrul diferitelor
tehnici de atac asupra memoriei cache.  % despre side channels & flush & reload

\include{3-continut}
\include{4-concluzii}
 
\printbibliography[heading=bibintoc]

\end{document}
