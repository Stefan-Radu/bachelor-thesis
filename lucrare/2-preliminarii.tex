\chapter{Preliminarii}

\section{Notiuni de Sisteme de Operare}

\subsection{Kernel}

Dupa cum sugereaza si numele, kernel-ul este componenta principala a unui
sistem de operare, care serveste drept interfata intre hardware si software.
Kernel-ul indeplineste patru roluri principale in cadrul sistemului de operare:

\begin{itemize}
  \setlength\itemsep{0em}
  \item gestionarea eficienta a memoriei de pe sistem
  \item programarea proceselor pe \emph{CPU}
  \item gestioneaza driverele de hardware, astfel actionand ca un mediator intre
    hardware si restul proceselor
  \item comunica cu restul proceselor prin intermediul unui interfete
    apelurilor de sistem (SCI)
\end{itemize}

Codul rulat in Kernel este izolat de restul codului de pe sistem. Acesta
intreprinde actiuni in mod privilegiat cu drepturi depline de acces asupra
hardware-ului. Codul obisnuit de pe sistem functioneaza in userland, si ruleaza
cu acces restrictionat asupra resurselor, avand acces la acestea doar prin
interfata sigura de comunicare cu Kernel mentionata mai sus (SCI)
\cite{kernel_redhat}.

% TODO trebuie structurata altfel partea de preliminarii. nu imi place deloc
% modul in care e organizat
% TODO fa zona asta mai coeziva (duplicate cred)

\subsection{Race Condition / Intrecere la rulare}

Un \emph{race-condition} apare la nivel de cod in momentul in care functionarea
corecta a unui program depinde de ordinea de executie sau de sincronizarea
temporala a mai multor fire de executie paralele (\emph{thread-uri}, sau
procese). In general aceste situatii apar in cazul in care firele de executie
vizeaza simultan o resursa comuna. Pentru evitarea bug-urilor in aceste
situatii, secventele operatii executate asupra resursei comune, numite
zone critice, trebuie executate intr-un mod reciproc exclusiv.

% TODO ref executie speculativa after redo
La nivel microarhitectural pot aparea \emph{race-condition-uri} in timpul
executiei speculative in urma carora pot aparea efecte
secundare exploatabile.

\subsection{IPC} 

\emph{Interprocess comunication}, sau \emph{IPC} face referire la mecanismele
puse la dispozitie de sistemul de operare pentru comunicarea intre procese si
gestionarea datelor comune. Principalele metode folosite in practica si amintite in
aceasta lucrare sunt:
\begin{itemize}
  \setlength\itemsep{0em}
  \item Fisiere. Comunicarea prin intermediul unor fisiere accesibile tuturor
    proceselor implicate.
  \item Semnale. Mesaje transmise de la un proces la altul, in general sub
    forma de instructiuni, corespunzand unui protocol stabilit anterior.
  \item Memorie partajata. Bloc de memorie la care au acces mai multe procese
    si prin intermediul caruia pot comunica.
\end{itemize}

\subsection{Exceptii de sistem}

O exceptie reprezinta o schimbare brusca in rularea programului ca raspuns la o
schimbare brusca in starea procesorului. Exemple de exceptii la nivel de
aplicatie ar fi: cereri de alocare a memoriei pe heap, cereri de input/output,
incercari de impartire cu $0$, incercari de accesare a memoriei in afara
limitelor impuse de memoria virtuala dedicata procesului, etc. Exceptiile se
impart in mai multe categorii: \emph{interrupts}, \emph{traps}, emph{faults},
\emph{aborts}. Categoria care va fi adusa in discutie in prezenta lucrare este
cea de-a treia, \emph{faults}. Defectele (\emph{faults}), sunt erori (posibil
recuperabile de catre sistemul de operare) cauzate de o aplicatie (eg.
accesarea unei zone de memorie asupra careia programul nu are drepturile
necesare --- \emph{segmentaion fault} ---, accesarea unei unor date care nu
sunt incarcate in memorie --- \emph{page fault}) \cite{exception_processes}.

\subsection{Procese}

Procesul reprezinta cea mai primitiva unitate de alocare a resurselor de sistem
si este o instanta activa a unui program \cite{processes}. Programul in acest
caz nu este nimic mai mult decat un fisier executabil stocat pe masina. Un
program nu poate rula decat in contextul unui proces, care consta in id-ul
procesului (\emph{PID}), spatiul de adrese (\emph{TEXT, DATA, STACK, HEAP, BSS,
etc}), starea procesului (starea registrilor), etc. \cite{exception_processes}.
Procesele sunt izolate intre ele, fiecare avand dedicat spatiul sau propriu de
adrese virtuale de memorie. Implicit procesele nu impart resurse intre ele, dar
pot comunica intre ele partajand resurse in mod intentionat cand acest obiectiv
este de dorit.


\subsection{Memorie Virtuala}

Procesoarele folosesc adrese virtuale de memorie si un mecanism de traducere a
acestora in adrese fizice pentru a asigura izolarea si separarea proceselor
intre ele. Fiecare zona de memorie virtuala este impartita in multiple pagini
(cea mai comuna dimensiune este de 4096 de bytes). Fiecare pagina virtuala este
mapata prin intermediul tabelelor de traducere a paginilor (\emph{translattion
table}), catre corespondentul fizic. In procesor exista un registru dedicat pentru 
retinerea tabelului de traducere utilizat la un moment dat si care se schimba la 
fiecare schimbare de context. In consecinta, ficare proces isi poate accesa doar 
zona sa virtuala de memorie. \\

Tabelele de traducere mai au si rolul de a asigura separarea intre zona de memorie
dedicata utilizatorului si zona de memorie dedicate kernel-ului in cadrul fiecarui
proces. In timp ce zona de memorie deicata utilizatorului poate fi accesata de 
aplicatia care ruleaza in procesul curet, zona de kernel poate fi accesata doar
prin intermediul unui utilizator privlegiat. Restrictiile acestea sunt precizate
in tablele de traducere, iar respectarea acestora este asigurata de sistemul de 
operare. Mai este important de notat faptul ca zona pentru kernel in general mapeaza
intreaga memorie fizica din cauza necesitatii de executie a diverselor operatii 
asupra acesteia (eg. scriere, sau citire de date).\\

Pe parcursul acestei lucrari vor fi expuse atacuri prin care limitele impuse de
tablele de traducere au fost ocolite, putandu-se accesa zona de kernel, si
implicit toata memoria fizica din postura unui utilizator neprivilegiat,
cat si accesul nepermis in zone de memorie ale altor procese, prin intermediul
unor pagini partajate.

\section{Notiuni de Arhitectura Sistemelor de Calcul}

\subsection{SMAP si SMEP}

SMAP si SMEP sunt doua caracteristici cu roluri in securizarea sistemelor prin
izolarea mai buna a Kernel-ului de spatiul utilizatorului (\emph{userland}).
Acestea sunt implementate in cadrul memoriei virtuale si activate prin setarea
bitilor corespunzatori ($20$ si $21$) din registrul \texttt{CR4} pe arhitectura
\emph{Intel}).

\emph{SMAP} este o caracteristica care presupune restrictia accesului asupra
anumitor zone de memorie din \emph{userland} in modul de executie Kernel. In
timp ce mecanismul de protectie este activat, incercarea de acces a zonelor
protejate va duce la declansarea unei exceptii. Rolul SMAP este de impiedica
programele malitioase din a manipula Kernelul sa acceseze instructiuni, sau
date nesigure din spatiul utilizatorului \cite{smap}.

SMEP este o caracteristica implementata cu scopul de a complementa SMAP. Are
rolul de a preveni executia neintentionata a unor fragmente de cod in spatiul 
user-ului, prin restrictiunatrea dreptului de executie asupra acestora. Diverse
atacuri precum cele de tipul \emph{Priviledge-Escalation} pot fi prevenite 
datorita acestor caracteristici.

\subsection{Out-of-order Execution \& Instructiuni Tranzitorii}

% TODO redo this

In trecut procesoarele executau instructiunile in ordinea in care acestea erau
preluate de la compilator, cate una pe rand. In multe situatii instructiuni mai
costisitoare blocau fluxul de executie, iar procesorul devenea partial inactiv.
Procesoarele moderne se folosesc de o serie de tehnici grupate sub umbrela
\emph{Out-of-order Execution}, introduse pentru prima data la mijlocul anilor
1990 \cite{what_is_speculative_execution}, in urma unui algoritm dezvoltat de 
Tomasulo in 1967 \cite{tomasulo1967} care permitea programarea dinamica
a ordinii instructiunilor si alocarea acestora pe mai multe unitati de executie
care ruleaza in paralel. Scopul acestei tehnici este utilizarea exhaustiva a
resurselor disponibile pe procesor, pentru cresterea performantei. Datorita
beneficiilor aduse, \emph{Out-of-order Execution} a devenit o caracteristica
indispensabila a sistemelor moderne de procesare.

Aceasta optimizare duce la situatii in care unele instructiuni executate
trebuie respinse, iar starea programului resetata la una anterioara (din cauza
decansarii unei exceptii in urma accesarii unei zone de memorie interzisa de
exemplu). Aceste tipuri de instructiuni numite in continuare \emph{Instructiuni
Tranzitorii} stau la baza atacului \emph{Meltdown} \cite{meltdown2018}.


\subsection{Branch Prediction \& Executie Speculativa}
\label{sec:branch_prediction}

\emph{Branch Processing Unit} (\emph{BPU}) din interiorul procesoarelor moderne
incearca sa prezica, in cazul unei ramificari a fluxului de executie (de
exemplu o structura decizionala -- \emph{if}), sau final de iteratie
(\emph{for, while}), ramura corecta care va fi urmata. In cazul in care fluxul
de executie stagneaza la un astfel punct de bifurcare (de exemplu, in
asteptarea incarcarii din memorie a valorii unei variabile), instructiunile
urmatoare se vor executa speculativ, urmand ramura prezisa de \emph{BPU}. Dupa
ce executia instructiunii care decide ramura corecta a executiei, rezultatele
obtinute speculativ sunt fie pastrate (caz in care se castiga timp de rulare pretios) fie respinse, caz in care se revine la o stare anterioara. \cite{spectre2019}. \\

Branch prediction are in general o acuratete foarte ridicata, chiar de peste
$95\%$ \cite{what_is_speculative_execution}, asadar executand speculativ s-au
obtinut imbunatatiri considerabile de performanta. Cu toate acestea, in
cazurile in care ramura de executie nu este prezisa corect, se vor executa
instructiuni care nu ar fi avut loc in cadrul executie secventiale,
\emph{in-order execution}. Bineinteles, aceste instructiuni vor fi
\emph{rolled-back}, iar rezultatul final va fi cel asteptat, dar la nivel
micro-arhitectural se pot observa si masura niste efecte secundare neprevazute
ale acestor instructiuni executate \emph{out-of-order}. Analizarea cu grija a
acestor efecte secundare sta la baza atacurilor de tip \emph{Spectre}
\cite{spectre2019}.

\subsection{CPU Cache}

Deoarece incarcarea valorilor din memoria RAM in CPU este foarte costisitoare,
in cadrul procesoarelor exista o ierarhie de zone de memorie foarte rapide,
separate in linii de dimensiuni mici (de obicei intre 16 si 128 de bytes), ce
poarta denumirea de emph{cache-uri} \cite{caching}. Dupa prima accesare a unei
adrese din memorie, valoarea obtinuta este retinuta in cache. Astfel, la
accesari ulterioare ale aceleiasi zone de memorie, timpul in care valoarea este
incarcata este redus semnificativ. In final, prin citiri repetate ale valorilor
din cache, se mascheaza incarcarea initiala din memorie, semnificativ mai
lenta, si se castiga timp pretios de executie. Cache-urile sunt de obicei
partajate intre nucleele unui procesor, optimizandu-se astfel si performanta
multi-core.

\section{Notiuni de Securitate}

\subsection{Atacuri asupra memoriei cache}\label{sec:atacuri_cache}

%TODO scoate exemplu din paper-ul cu missing the cache for fun and profit

Deoarece memoria cache este mult mai rapida, prin intermediul unui ceas de mare precizie
putem distinge intre accesare din memorie si accesarea din \emph{cache} a unei variabile.
Sa consideram urmatoarea secventa de cod:

\begin{lstlisting}[language=c]
  uint32_t value = 10;
  addr = &secret; // adresa secretului
  /* valoare irelevanta.
     folosita ca referinta pentru cronometru
     static -> important pentru a preveni optimizarea
     in moduri nedorite */
  static int junk = 0; 

  time = __rdtscp(&junk);
  junk = *addr; \label{code:junk_flush_reload}
  // prima accesare din memorie
  memory_time = __rdtscp(&junk) - time;

  addr = &value;
  time = __rdtscp(&junk);
  junk = *addr;
  // a doua accesare din cache
  cache_time = __rdtscp(&junk) - time;
\end{lstlisting}

% TODO listing-ul acesta cred ca trebuie mutat la implementarea atacului sau idk

Timpul de accesare al valorii corespunzatoare variabilei \texttt{value} poate
fi calculat utilizand instructiunea \texttt{\_\_rdtscp} specifica procesoarelor
Intel. Aceasta permite citirea \emph{time-stamp counter-ului} din procesor
\cite{rdtscp}. Prin doua masuratori ce incadreaza dereferentierea pointer-ului
catre \texttt{value}, putem masura numarul de ciclii de procesor necesari
operatiei. Repetand experimentul de $10000$ de ori si calculand media
timpului de acces pentru fiecare caz, se obtin urmatoarele rezultate:

\begin{itemize}
  \setlength\itemsep{0em}
  \item incarcarea din memorie dureaza aproximativ $250$ de ciclii si poarta
    numele de \emph{cache miss}
  \item incarcarea din cache dureaza aproximativ $23$ de ciclii si poarta
    numele de \emph{cache hit}
\end{itemize}

Aceste diferenta masurabile sunt exploatate in cadrul diferitelor
tehnici de atac asupra memoriei cache, printre care si \emph{FLUSH and RELOAD}
care va fi discutat in continuare.

\subsubsection{Observatie}

% TODO as putea face o sectiune separata de preliminarii in care sa mentionez
% specificatiile sistemului pe car s-au facut toate testele
Numarul de ciclii de procesor necesari executiei unui set de instructiuni
difera in functie de sistem. Rezultatele ilustrate anterior sunt specifice unui
sistem ce ruleaza o versiune actualizata a Kernelului Linux in data de
22.05.2022 (\texttt{Linux 5.17.9-arch1-1 x86\_64}), cu un procesor al
producatorului \emph{Intel}, modelul \texttt{i5-8250U} (mai multe specificatii
pe site-ul producatorului \cite{i5_8250U}), cu 8GB de memorie RAM tip DDR3.


% TODO ia in considerare sa mentionezi si din articolul cu cache miss for fun and profit
% ori aici in preliminarii, ori intr-o sectiune dedicata lui. -> !!!! VB. CU IROFTI

\subsubsection{FLUSH and RELOAD}\label{sec:flush_reload}

O practica comuna de reducere a memoriei utilizate este partajarea intre
procese a unor pagini comune cu drepturi exclusive de citire
(\emph{read-only}). \emph{FLUSH and RELOAD} este una dintre tehnicile
documentate de atac asupra memoriei cache. Scenariul descris in lucrarea de
cercetare in care a fost introdus atacul este acela al unei victime si al unui
spion care impart o zona partajata de memorie. Spionul se foloseste de
instructiunea \texttt{clflush} care invalideaza liniile aferente unei zone de
memorie din toata ierarhia cache-ului din procesor \cite{clflush}, iar apoi
asteapta o perioada scurta de timp. In final verifica daca la accesarea zonei
respective obtine un \emph{cache hit} sau un \emph{cache miss}, astfel afland
daca victima a accesat sau nu in fereastra respectiva de timp, zona de memorie
urmarita. Repetand experimentul, s-au putut extrage informatii suficiente
pentru realizarea unor atacuri de succes asupra implementarii de la vremea
respectiva a unor algoritmi criptografici (\emph{OpenSSL}, \emph{AES}),
monitorizarea activitatii unui utilizator, etc. \cite{flush_reload}.

\subsubsection{Alte tipuri de atacuri asupra memoriei cache}

\emph{FLUSH and FLUSH} se aseamana cu \emph{FLUSH and RELOAD}, diferenta 
constand in faptul ca spionul in loc de a masura timpul de acces a zonei
tinta, va apela iarasi \texttt{clflush}. Executia mai rapida va corespunde
unui \emph{cache miss}, iar cea mai rapida unui \emph{cache hit} \cite{cache_attacks}.

\emph{EVICT and RELOAD} foloseste un \emph{eviction set} pentru a elimina din
cache zona de memorie tinta. Apoi, pentru masurarea timpului se procedeaza 
identic ca la \emph{FLUSH and RELOAD}. Tehnica se aseamana in eficienta cu 
cele mentionate anterior \cite{cache_attacks}.

\emph{PRIME + PROBE} consta intr-o abordare diferita. Atacatorul umple toata
zona partajata din cache (\emph{PRIME}). Victima va elimina valori in carcate
de atacator in cache in timp ce ruleaza (\emph{evict}). Atacatorul va masura 
apoi timpul de acces pentru toata zona de cache (\emph{probe}). In cazul in care observa
un \emph{cache hit}, constata ca victima a accesat zona respectiva \cite{cache_attacks}.

\subsection{Covert-channel}

Utilizarea tehnicilor descrise anterior pentru extragerea diverselor
informatii, fac ca memoria cache sa devina un \emph{canal secundar} de
comunicare(\emph{side-channel}), iar atacurile poarta numele de
\emph{side-channel attacks}. In momentul in care atacatorul controleaza atat
modul in care este indus efectul secundar cat si modul in care este masurat,
avem in discutie o subcategorie a atacurilor pe \emph{canal secundar}, mai
preci atacuri pe \emph{canal ascuns} (\emph{covert-channel}).
 
%TODO e scris ca naiba. vezi pe wikipedia ca zice bine

\subsection{Atacuri Speculative}

Atacurile Speculative se bazeaza pe exploatarea tehnicii de optimizare numita
\emph{Executie Speculativa} care, datorita avantajelor aduse in performata,
este utilizata in prezent de majoritatea procesoarelor folosite in prezent.
Atacurite se folosesc de aceasta optimizare pentru a produce intentionat efecte
secundare masurabile, cu scopul de a accesa date in mod neautoriazat, prin
intermediul unor canale secundare (\emph{side channeles}) sau ascunse
(\emph{covert channels}). In continuarea acestei lucrari voi discuta
particularitatile si implicatiile catorva atacuri din aceasta clasa care au
avut un impact semnificativ asupra industriei in ultimii ani.

\subsection{ROP}\label{sec:rop}

\emph{ROP} \cite{shacham2007geometry} este o tehnica prin care un atacator care
reuseste sa deturneze fluxul normal de instructiuni, poate sa manipuleze
victima in realizarea unor actiuni complexe. In acest scop, atacatorul executa
in lant sevente reduse ca dimensiune de instructiuni masina numite
\emph{gadget-uri}. Gaget-urile sunt prezente si identificate de atacator in
codul sursa al victimei si se aseamana prin faptul ca realizeaza operatii
oarecare inaintea executarii unei instrutiuni (sau set de instructiuni) de tip
\emph{return}. Daca un atacator poate prelua controlul
\emph{stack-pointer-ului}, atunci poate redirectiona foluxul de instructiunui
catre un gadget special ales, care la randul lui va ridirectiona fluxul catre
alt gadget. S-a demonstrat ca un set restrans de gadget-uri poate fi echivalent
cu un limbaj Turing-Complete \cite{homescu2012microgadgets}. Pe idei preluate
din \emph{ROP} se bazeaza variante ale clasei de atacuri \emph{Spectre}, care
apeleaza in mod speculativ \emph{gadget-uri} din spatiul de meomorie al
victimei.
