\begin{abstractpage}

\begin{abstract}{romanian}

  Computerele moderne folosesc tehnici de optimizare precum \emph{execuție
  out-of-order} și \emph{branch prediction}. \emph{Meltdown} și \emph{Spectre}
  sunt două atacuri care exploatează efectele secundare apărute la nivel
  microarhitectural în urma optimizărilor menționate. Prin intermediul acestora
  un atacator poate citi date private din zone arbitrare din memorie, fară
  privilegii și fară a exploata niciun bug de natură software. \emph{Intel},
  \emph{AMD} și \emph{ARM} au fost forțate în urma divulgării acestor atacuri
  să își schimbe designul procesoarelor în încercarea de a mitiga
  vulnerabilitățile la nivel hardware. În ciuda soluțiilor implementate, la
  jumătatea anului 2022, \emph{Spectre} afectează în continure majoritatea
  computerelor din lumea întreagă și rămâne un pericol pentru utilizatori și un
  subiect de mare interes pentru cercetători. În această lucrare vor fi
  prezentate particularitățile celor două atacuri, și o implementare
  demonstrativă a unui atac de tip \emph{Spectre}.

\end{abstract}

\begin{abstract}{english}
  
  Modern computers are equiped with features such as \emph{out-of-order
  execution} and \emph{branch prediction}, which are used to reduce CPU idel
  time and improve performance. \emph{Meltdown} and \emph{Spectre} are to cyber
  attacks that exploit microarhitectural side-effects which apper as a result
  of such optimization techiniques being used. An attacker can read private
  data of the vicim at arbitrary locations in memory, without exploiting any
  software bug. \emph{Intel}, \emph{AMD} and \emph{ARM} were forced to redesign
  their CPUs in order to migiate the risks posed by \emph{Meltdown} and
  \emph{Spectre}. Despite deployed mitigations, in the second half of 2022,
  most computers in the world are vulnerable to variations of \emph{Spectre}
  attacks, billions of users begin at risk. This class of attacks remains a
  subject of great interest for researchers in the field of security. In this
  work, the technicalities and implications of both attacks will be covered.
  Moreover, a proof of concept for a \emph{Spectre} attack will be presented.

\end{abstract}

\end{abstractpage}
